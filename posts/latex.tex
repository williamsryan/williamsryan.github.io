---
title: Rendering LaTeX in Hakyll
date: 2023-11-14
toc: true
---

\usepackage{listings}
% \usepackage{minted}
\usepackage{hyperref}

\newtheorem{theorem}{Theorem}
\newtheorem{corollary}[theorem]{Corollary}
\newtheorem{lemma}[theorem]{Lemma}
\theoremstyle{definition}
\newtheorem{definition}[theorem]{Definition}
\theoremstyle{remark}
\newtheorem{remark}{Remark}

\title{Rendering LaTeX in Hakyll}

\section{What is Hakyll?}
\href{https://jaspervdj.be/hakyll/}{Hakyll} is a Haskell library for generating static sites, mostly aimed at small-to-medium sites and personal blogs.
It is written in a very configurable way and uses an \href{https://xmonad.org/}{xmonad}-like DSL for configuration. \\
\\
When I first started putting together this site, I wanted more flexibility for rendering the Markdown posts in HTML.
For my particular needs, LaTeX offered everything I wanted to show here.
Now, this site as you see it uses Markdown and Hakyll-specific formatting \textit{within} LaTeX.

\section{Rendering Markdown in Hakyll}
As Hakyll is integrated with \href{https://pandoc.org/}{pandoc}, it is capable of rendering Markdown with proper formatting in HTML.
For example, to produce header text \footnote{which is used as our table of contents when setting \texttt{toc} to \texttt{true}} we can simply use:
\begin{verbatim}
# Header Text
## Subheader Text
\end{verbatim}
Similarly, we can produce code blocks with syntax highlighting:
\begin{verbatim}
```haskell
main :: IO ()
main = putStrLn "Hello, world!"
```
\end{verbatim}
which produces:
\begin{lstlisting}[language=haskell]
main :: IO ()
main = putStrLn "Hello, world!"
\end{lstlisting}

\subsection{Compiling to HTML}
Using Markdown files on a Hakyll site is as simple as placing them in a directory, say \texttt{posts}, and then compiling them to HTML like so:
\begin{lstlisting}[language=haskell]
match "posts/*.md" $ do
    route $ setExtension "html"
    compile $ pandocCompiler
        >>= relativizeUrls
\end{lstlisting}

\section{Rendering LaTeX in Hakyll}
While TeX files are already supported by Hakyll as it uses pandoc, the compilation process is a bit different than the case of Markdown.
We can use support files with our LaTeX, like class or style files\footnote{we can even use a \texttt{bib} file}, however, in its simplest form we can start by using a TeX file like so:
\begin{verbatim}
\title{Some Title}
\section{Some Section}
Some text.
\end{verbatim}

\subsection{Special Characters}
As we are using LaTeX, that provides us with a simple way of defining mathematical expressions, among other things.
Next, we show some examples of how to use native LaTeX commands in Hakyll:
\begin{verbatim}
\newtheorem{theorem}{Theorem}
\newtheorem{corollary}[theorem]{Corollary}
\newtheorem{lemma}[theorem]{Lemma}
\theoremstyle{definition}
\newtheorem{definition}[theorem]{Definition}
\theoremstyle{remark}
\newtheorem{remark}{Remark}
\end{verbatim}
which can be used like so:
\begin{verbatim}
\begin{definition}[right-angled triangles]
    A \emph{right-angled triangle} is a triangle whose sides of length~\(a\), \(b\) and~\(c\), in some permutation of order, satisfies \(a^2+b^2=c^2\).
\end{definition}
\end{verbatim}
which produces:
\begin{quote}
\begin{definition}[right-angled triangles]
    A \emph{right-angled triangle} is a triangle whose sides of length~\(a\), \(b\) and~\(c\), in some permutation of order, satisfies \(a^2+b^2=c^2\).
\end{definition}
\end{quote}

Similarly:
\begin{verbatim}
\begin{proof}
    This lemma follows from \cref{def:tri} since \(3^2+4^2=9+16=25=5^2\).
\end{proof}
\end{verbatim}
produces:
\begin{quote}
\begin{proof}
    This lemma follows from \cref{def:tri} since \(3^2+4^2=9+16=25=5^2\).
\end{proof}
\end{quote}

\subsection{Compiling to HTML}
This is where we need to handle things a bit differently.
Using Hakyll's pandoc integration, we can compile our TeX files to HTML like so:
\begin{lstlisting}[language=haskell]
match "*/*.tex" $ do
    route   $ constRoute "*/index.html"
    compile $ getResourceString
        >>= withItemBody (unixFilter "pandoc" ["-f", "latex", "-t", "html5","--mathjax"])
        >>= relativizeUrls
\end{lstlisting}