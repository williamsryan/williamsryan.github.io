---
title: "Coq Cheatsheet: Tools and Tactics for Goal Solving"
date: 2023-12-19
toc: true
---
\documentclass{article}

\usepackage{amsmath}
\usepackage{listings}
\usepackage{minted}

\begin{document}

\title{Coq Cheatsheet: Tools and Tactics for Goal Solving}

\section{Introduction}
This cheatsheet provides a quick reference to various tools and tactics in Coq, accompanied by examples to illustrate their usage in solving goals.

\section{Basic Tactics}
\subsection{\texttt{intros}}
\textbf{Usage:} Introduce hypotheses and variables into the context. \\
\textbf{Example:} Proving symmetry in equality.
\begin{lstlisting}[language=haskell]
Theorem symmetry_eq : forall a b : nat, a = b -> b = a.
Proof. intros a b H. apply H. Qed.
\end{lstlisting}
This theorem states that if \( a = b \), then \( b = a \). The \texttt{intros} tactic introduces variables \texttt{a}, \texttt{b}, and hypothesis \texttt{H} into the proof context.

\subsection{\texttt{apply}}
\textbf{Usage:} Apply a theorem, lemma, or hypothesis. \\
\textbf{Example:} Using transitivity of equality.
\begin{lstlisting}[language=haskell]
Theorem trans_eq : forall a b c : nat, (a = b) -> (b = c) -> (a = c).
Proof. intros a b c H1 H2. apply H1. apply H2. Qed.
\end{lstlisting}
Here, \texttt{apply} is used to use the hypotheses \txttt{H1} and \txttt{H2} in the proof.

\subsection{\texttt{rewrite}}
\textbf{Usage:} Rewrite a goal using a hypothesis or theorem. \\
\textbf{Example:} Demonstrating rewriting.
\begin{lstlisting}[language=haskell]
Theorem rewrite_example : forall a b c : nat, (a = b) -> (a + c = b + c).
Proof. intros a b c H. rewrite -> H. reflexivity. Qed.
\end{lstlisting}
\texttt{rewrite -> H} rewrites \texttt{a} in the goal with \texttt{b}, using hypothesis \texttt{H}.

\subsection{\texttt{reflexivity}}
\textbf{Usage:} Prove a goal of the form \texttt{a = a}. \\
\textbf{Example:}
\begin{lstlisting}[language=haskell]
    Lemma use_reflexivity:
    (* Simple theorem, x always equals x *)
    forall x: Set, x = x.
Proof.
    (* Introduce variables/hypotheses *)
    intro.
    reflexivity.
Qed.
\end{lstlisting}

\subsection{\texttt{assumption}}
\textbf{Usage:} When the goal is already in the context, you can use \texttt{assumption} to prove it.\\
\textbf{Example:}
\begin{lstlisting}[language=haskell]
    Lemma p_implies_p : forall P : Prop,
    (* P implies P is always true *)
    P -> P.
Proof.
    (* We can specify names for introduced variables/hypotheses *)
    intros P P_holds.
\end{lstlisting}
After introducing this hypothesis, \texttt{P_holds} (stating that \texttt{P} is true) into the context, we can use \texttt{assumption} to complete the proof:
\begin{lstlisting}[language=haskell]
    Lemma p_implies_p : forall P : Prop,
    P -> P.
Proof.
    intros P P_holds.
    assumption.
Qed.
\end{lstlisting}

\section{Logical Tactics}
\subsection{\texttt{split}}
\textbf{Usage:} Split a conjunction into two separate goals. \\
\textbf{Example:}
\begin{lstlisting}[language=haskell]
split.
\end{lstlisting}

\subsection{\texttt{left / right}}
\textbf{Usage:} Prove a disjunction by proving one of its parts. \\
\textbf{Example:}
\begin{lstlisting}[language=haskell]
left.
\end{lstlisting}

\subsection{\texttt{exists}}
\textbf{Usage:} Provide a witness for an existential quantifier. \\
\textbf{Example:}
\begin{lstlisting}[language=haskell]
exists x.
\end{lstlisting}

\section{Advanced Tactics}
\subsection{\texttt{induction}}
\textbf{Usage:} Apply induction on a variable. \\
\textbf{Example:}
\begin{lstlisting}[language=haskell]
induction n.
\end{lstlisting}

\subsection{\texttt{inversion}}
\textbf{Usage:} Derive information from equality of inductive types. \\
\textbf{Example:}
\begin{lstlisting}[language=haskell]
inversion H.
\end{lstlisting}

\subsection{\texttt{destruct}}
\textbf{Usage:} Case analysis on a variable or hypothesis. \\
\textbf{Example:}
\begin{lstlisting}[language=haskell]
destruct n.
\end{lstlisting}

\section{Conclusion}
This cheatsheet offers a glimpse into the powerful tactics at your disposal in Coq. Experimenting with these tactics will deepen your understanding and proficiency in proof construction.

\end{document}